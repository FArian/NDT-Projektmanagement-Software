\chapter{Einleitung}
\label{sec:einleitung}
Bevor ich das Studium der Informatik begann, arbeitete ich im Bereich der RT-Werkstoffprüfung.\\(Siehe Teil: \ref{subsec:ndt}) Meine Aufgabe war es, als Beauftragter für den Bereich Strahlenschutz/Health Physics(\textcolor{red}{HPS})(Siehe Teil:\ref{sec:strahlenschutzbeauftragter}) dafür zu sorgen, dass die Mitarbeiter des Unternehmens, für das ich tätig war, vor Strahlung geschützt werden.
Leider is es in der Praxis nicht selten passiert, dass Mitarbeiter sich nicht an die Bestimmungen
u.a. unverzügliche und vollständige Übermittlung von Daten, die der Auswertung zu ihrem persönlichen, physischen Schutz dienen sollen, hielten.
Somit setzten sich Mitarbeiter der Strahlenbelastung aus.
Das hätte verhindert werden können, wenn Sicherheitsmaßnahmen konsequenter umgesetzt und kontrolliert gewesen wären.
Daher entstand bei mir die Idee, eine Software zu entwickeln, die es einzelnen Mitgliedern eines Teams ermöglicht, relevante und angeforderte Daten schnell, einfach und transparent an die zuständigen Beauftragten für Health Physics (HPS) in der Zentrale zu übermitteln.
Zwar gibt es bereits Software, die einzelne Aufgaben im Bereich Technik erfüllen, jedoch nicht im Bereich Sicherheit und nicht zusammengeführt in einer bedienungsfreundlichen App.
Daher ist es Ziel dieser Arbeit eine App, die verschiedene Funktionen zur Verbesserung der Datenübermittlung zum Schutz der physischen Gesundheit und auch zur Vereinfachung der organisatorischen Abläufe und Materialbeschaffung im Projekt vereint, zu entwickeln.  
