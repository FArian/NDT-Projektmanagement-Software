\addsec{Zusammenfassung / Abstract}
\label{sec:zusammenfassung}

Radioaktive Strahlung führt zu Strahlenschäden in den Körperzellen. Es gibt zwei Arten von Schäden: Frühschäden und Spätschäden.
Die Frühschäden treten nur bei hohen Dosen auf, dann aber schnell je nach Dosis innerhalb von Stunden bis Tagen. Spätschäden können hingegen noch Jahrzehnte nach einer weniger starken Strahlenbelastung auftreten. Das Krankheitsbild ist in beiden Fällen unterschiedlich und hängt von der Art der Strahlenbelastung, dem Gesundheitszustand des Patienten, seiner Anfälligkeit für Strahlung, seiner genetischen Disposition für Krebs und eventuellen Gegenmaßnahmen ab.
Die Strahlenbelastung wird in Sievert (Sv) gemessen. Diese Einheit gibt die biologische Wirksamkeit radioaktiver Strahlung an. Sie ermittelt sich aus der Art, Stärke und Dauer der Strahlung sowie dem betroffenen Gewebe.
{\rowcolors{3}{yellow}{green}
\begin{tabular}{ |p{3cm}||p{2cm}|p{2cm}|p{2cm}|p{2cm}|p{2cm}| }
 \hline
 \multicolumn{6}{|c|}{ Half-Value Layer, mm (inch)} \\
 \hline
 Source& Concrete &Steel &Lead&Tungsten&Uranium\\
  \hline
 Iridium-192 & 44.5 (1.75)&12.7 (0.5)&4.8 (0.19)&3.3 (0.13)&2.8 (0.11)\\
  \hline
 Cobalt-60   &60.5 (2.38)&21.6 (0.85)&12.5 (0.49)&7.9 (0.31)&6.9 (0.27)\\
 \hline
\end{tabular} 
Verschiedene gemeinnützige Unternehmen sowie Atomenergie-Behörde versuchen, den negativen Auswirkungen der Strahlung der Radioaktivität zu begegnen.
Zur Verbesserung werden Bildung,Schulung und Technologie benötigt. Doch in der heutigen Industrie verringert sich die Zahl der Radiographen, die Sicherheitsmaßnahmen richtig berücksichtigen, wodurch eine Strahlenschäden entsteht.
Auch geringe Dosen an Radioaktivität führen zu Schäden in Zellen. Die körpereigenen Reparaturmechanismen können mit diesen allerdings gut umgehen.
Solange keine wesentlich höheren Dosen als die natürliche Strahlenbelastung auftreten, besteht keine nachweisbare Gesundheitsgefährdung.
Zwar gibt es nach heutigem wissenschaftlichem Kenntnisstand keine untere Grenze, ab der Radioaktivität gänzlich ungefährlich ist.
Um die Strahlenschäden vermeiden zu können, ist es notwendig, dass die Sicherheitsinformationen rechtzeitig und täglich dokumentiert und an health physics Society (HPS) informiert werden.

\minisec{Abstract}
\label{abstract}
