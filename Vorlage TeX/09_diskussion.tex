\chapter{Diskussion}
\label{sec:diskussion}

\section{Zusammenfassung}
\label{sec:überschrift}

Radioaktive Strahlung führt zu Strahlenschäden in den Körperzellen.
Es gibt zwei Arten von Schäden: Frühschäden und Spätschäden.
Die Frühschäden treten nur bei hohen Dosen auf, dann aber schnell je nach Dosis innerhalb von Stunden bis Tagen. Spätschäden können hingegen noch Jahrzehnte nach einer weniger starken Strahlenbelastung auftreten. Das Krankheitsbild ist in beiden Fällen unterschiedlich und hängt von der Art der Strahlenbelastung, dem Gesundheitszustand des Patienten, seiner Anfälligkeit für Strahlung, seiner genetischen Disposition für Krebs und eventuellen Gegenmaßnahmen ab.
Die Strahlenbelastung wird in Sievert (Sv) gemessen. Diese Einheit gibt die biologische Wirksamkeit radioaktiver Strahlung an. Sie ermittelt sich aus der Art, Stärke und Dauer der Strahlung sowie dem betroffenen Gewebe.\\
%%==========================Table====================================================
{\rowcolors{3}{green}{yellow}
\begin{tabular}{ |p{6.5cm}||p{9cm}| }
 \hline
 \multicolumn{2}{|c|}{\href{https://scilogs.spektrum.de/atommuell-debatte}
 {\textcolor{blue}{Frühschäden}}}\\
 \hline
 Strahlendosis& Auswirkungen und Symptome\\
  \hline
 1 bis 5 Milli-Sievert (mSv) pro Jahr&Durchschnittliche jährliche radioaktive Belastung.\\
  \hline
 ab 100 mSv (0,1 Sv)&Leicht erhöhtes Krebsrisiko nachweisbar.\\
 \hline
0,5 bis 1 Sv&Kopfschmerzen,Übelkeit, Abgeschlagenheit
                            und erhöhtes Infektionsrisiko.\\
 \hline
 1 bis 2 Sv&ab 1 Sv: Akute Strahlenkrankheit:10\% Todesfälle nach 30 Tagen,
             mittlere Übelkeit, Erbrechen, verzögerte Wundheilung, 
                     Ermüdung, Verlust von weißen Blutkörperchen, 
                     stark gestiegenes Infektionsrisiko.\\
 \hline
 2 bis 3 Sv&35\% Todesfälle nach 30 Tagen:
                       schwere Übelkeit, häufiges Erbrechen,
                        massiver Verlust von weißen Blutkörperchen.\\
  \hline
3 bis 4 Sv&50\% Todesfälle nach 30 Tagen:
                        zusätzliche Symptome Durchfall, 
                        unkontrollierte Blutungen im Mund, 
                        unter der Haut und in den Nieren.\\
  \hline
4 bis 6 Sv&Bis zu 90\% Todesfälle nach 30 Tagen:
                       Symptome wie oben, aber verstärkt.
                       Todesursache nach wenigen Wochen durch Infektionen und Blutungen.\\
 \hline
  über 6 Sv&100\% Todesfälle nach 14 Tagen:
            Sterbephase mit schnellem Zelltod.\\
 \hline

\end{tabular} \\
%===============================ENDE TABLE ==================================

Verschiedene gemeinnützige Unternehmen sowie Atomenergie-Behörden versuchen, den negativen Auswirkungen der Strahlung der Radioaktivität zu begegnen.
Zur Verbesserung der Sicherheit werden Bildung, Schulung und Technologie benötigt.
Doch in der heutigen Industrie verringert sich die Zahl der Radiographen, die die Sicherheitsmaßnahmen in Kauf nehmen, wodurch Strahlenschäden entstehen.
Auch geringe Dosen an Radioaktivität führen zu Schäden in Zellen. Die körpereigenen Reparaturmechanismen können mit diesen allerdings gut umgehen.
Solange keine wesentlich höheren Dosen als die natürliche Strahlenbelastung auftreten, besteht keine nachweisbare Gesundheitsgefährdung.
Zwar gibt es nach heutigem wissenschaftlichen Kenntnisstand keine untere Grenze, ab der Radioaktivität gänzlich ungefährlich ist.
Um die Strahlenschäden vermeiden zu können, ist es notwendig, dass die Sicherheitsinformationen rechtzeitig und täglich dokumentiert und an den Verantwortlichen für health physics Safety (HPS) mitgeteilt werden.Dies erfolgt häufig nicht rechtzeitig in der Praxis.
Aufgrund der Entfernung der radioaktiven Projekte werden oft falsche oder verzögerte Informationen an den health physics Safety (HPS) gesendet.
Aufgrund der Erleichterung der Dokumentation und auch des besseren Managements wurde die Software vor allem zur Erhöhung der Sicherheit entwickelt.  
Weil jeder mindestens ein Android-basiertes Handy besitzt, das Software erst für Android-Betriebssysteme programmiert.
Die erste Version der Software bietet viel an, sodass neben der Sicherheit und des besseren Managements weitere Funktionen wie Timing, Report, Sicherheitsdokumentation ermöglicht werden.
Eine ähnliche Software habe ich nicht gefunden, jedoch gibt es welche, die nur eine bestimmte Funktion anbieten können.
In der von mir erstellten Software werden die verschiedenen für den Strahlenschutz relevanten Funktionen vereint.
Diese Software kann aber auch auf andere Typen-Bereiche in der Werkstoffprüfung wie MT,UT etc.
angepasst werden.
Als Bachelorarbeit lege ich nun die erste Version der Software vor.
In der Anwendung im realen Arbeitsprozess wird diese Version erweitert und verbessert werden.
 
\section{Quellen}
\label{sec:quellen}

\begin{itemize}
\item  The American Safety For Nondestructive Testing (2018).\\Verfügbar unter: \url{https://www.asnt.org/Home}
\item Werkstoffprüfung (2018).\\
 Verfügbar unter: \url{https://de.wikipedia.org/wiki/Werkstoffprüfung}
\item Strahlenschutzbereich (2018)\\
\url{https://de.wikipedia.org/wiki/Strahlenschutzbereich}
\url{http://www.bfs.de/bfs/glossar.html} [14.02.2012]
\item Radiographic Testing (2018). Verfügbar unter:\\
\url{https://inspectioneering.com/tag/radiography } [18.02.2018]
\item Strahlenschutzbereich(2017).\\
 Verfügbar unter: \url{https://de.wikipedia.org/wiki/Strahlenschutzbereich}\\

\item Bundesamt für Strahlenschutz. (2012) Glossar.Verfügbar unter:\\ \url{https://web.archive.org/web/20120502144722}\\
\item Ghiassi-Nejad, M., Katouzi, M. (2004).\\
Hefazat dar barbare aschaeh (Schutz gegen radioaktive Strahlung). 2. Band. 8. Auflage.\\
Teheran: Iranische Nationalbibliothek.
\item NDT Resource Center. Verfügbar unter:\\ \url{https://www.nde-ed.org/index_flash.htm}
 \item Atommüll.\\
 Verfügbar unter: \url{https://scilogs.spektrum.de/atommuell-debatte}
\end{itemize}

